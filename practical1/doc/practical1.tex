%%%%%%%%%%%%%%%%%%%%%%%%%%%%%%%%%%%%%%%%%
% Journal Article
% Compiler Design
%
% Gahan M. Saraiya
% 18MCEC10
%
%%%%%%%%%%%%%%%%%%%%%%%%%%%%%%%%%%%%%%%%%
%----------------------------------------------------------------------------------------
%       PACKAGES AND OTHER DOCUMENT CONFIGURATIONS
%----------------------------------------------------------------------------------------
\documentclass[paper=letter, fontsize=12pt]{article}
\usepackage[english]{babel} % English language/hyphenation
\usepackage{amsmath,amsfonts,amsthm} % Math packages
\usepackage[utf8]{inputenc}
\usepackage{float}
\usepackage{lipsum} % Package to generate dummy text throughout this template
\usepackage{blindtext}
\usepackage{graphicx} 
\usepackage{caption}
\usepackage{subcaption}
\usepackage[sc]{mathpazo} % Use the Palatino font
\usepackage[T1]{fontenc} % Use 8-bit encoding that has 256 glyphs
\usepackage{bbding}  % to use custom itemize font
\linespread{1.05} % Line spacing - Palatino needs more space between lines
\usepackage{microtype} % Slightly tweak font spacing for aesthetics
\usepackage[hmarginratio=1:1,top=32mm,columnsep=20pt]{geometry} % Document margins
\usepackage{multicol} % Used for the two-column layout of the document
%\usepackage[hang, small,labelfont=bf,up,textfont=it,up]{caption} % Custom captions under/above floats in tables or figures
\usepackage{booktabs} % Horizontal rules in tables
\usepackage{float} % Required for tables and figures in the multi-column environment - they need to be placed in specific locations with the [H] (e.g. \begin{table}[H])
\usepackage{hyperref} % For hyperlinks in the PDF
\usepackage{lettrine} % The lettrine is the first enlarged letter at the beginning of the text
\usepackage{paralist} % Used for the compactitem environment which makes bullet points with less space between them
\usepackage{abstract} % Allows abstract customization
\renewcommand{\abstractnamefont}{\normalfont\bfseries} % Set the "Abstract" text to bold
\renewcommand{\abstracttextfont}{\normalfont\small\itshape} % Set the abstract itself to small italic text
\usepackage{titlesec} % Allows customization of titles

\renewcommand\thesection{\Roman{section}} % Roman numerals for the sections
\renewcommand\thesubsection{\Roman{subsection}} % Roman numerals for subsections

\date{}
\hypersetup{
	colorlinks=true,
	linkcolor=blue,
	filecolor=magenta,      
	urlcolor=cyan,
	pdfauthor={Gahan Saraiya},
	pdfcreator={Gahan Saraiya},
	pdfproducer={Gahan Saraiya},
}

\usepackage{makecell}
\usepackage{longtable}

\newcommand*\tick{\item[\Checkmark]}
\newcommand*\good{\CheckmarkBold}
\newcommand*\arrow{\item[$\Rightarrow$]}
\newcommand*\fail{\item[\XSolidBrush]}
\newcommand*\bad{\XSolidBrush}
\usepackage{minted} % for highlighting code sytax
\usepackage{xcolor} % for highlighting code sytax
\definecolor{LightGray}{gray}{0.9}

\setminted[text]{
	frame=lines, 
	breaklines,
	baselinestretch=1.2,
	bgcolor=LightGray,
	%	fontsize=\small
}
\setminted[sql]{
	frame=lines, 
	breaklines,
	baselinestretch=1.2,
	bgcolor=LightGray,
	%	fontsize=\small
}

\setminted[python]{
	frame=lines, 
	breaklines, 
	linenos,
	baselinestretch=1.2,
	%	bgcolor=LightGray,
	%	fontsize=\small
}
\titleformat{\section}[block]{\large\scshape\centering}{\thesection.}{1em}{} % Change the look of the section titles
\titleformat{\subsection}[block]{\large}{\thesubsection.}{1em}{} % Change the look of the section titles
\newcommand{\horrule}[1]{\rule{\linewidth}{#1}} % Create horizontal rule command with 1 argument of height
\usepackage{fancyhdr} % Headers and footers
\pagestyle{fancy} % All pages have headers and footers
\fancyhead{} % Blank out the default header
\fancyfoot{} % Blank out the default footer
%----------------------------------------------------------------------------------------
%       DATE FORMAT
%----------------------------------------------------------------------------------------
\usepackage{datetime}
\newdateformat{monthyeardate}{\monthname[\THEMONTH], \THEYEAR}
%----------------------------------------------------------------------------------------

%----------------------------------------------------------------------------------------
%       TITLE SECTION
%----------------------------------------------------------------------------------------
\title{\vspace{-15mm}\fontsize{24pt}{10pt}\selectfont\textbf{Practical 1: List of the compiler Designer tools and write its comparative analysis}} % Article title
\author{
	\large
	{\textsc{Gahan Saraiya (18MCEC10)}}\\[2mm]
	%\thanks{A thank you or further information}\\ % Your name
	\normalsize \href{mailto:18mcec10@nirmauni.ac.in}{18mcec10@nirmauni.ac.in}\\[2mm] % Your email address
}
\date{}
\hypersetup{
	colorlinks=true,
	linkcolor=blue,
	filecolor=magenta,      
	urlcolor=cyan,
	pdfauthor={Gahan Saraiya},
	pdfcreator={Gahan Saraiya},
	pdfproducer={Gahan Saraiya},
}
%----------------------------------------------------------------------------------------

%----------------------------------------------------------------------------------------
%       SET HEADER AND FOOTER
%----------------------------------------------------------------------------------------
\newcommand\theauthor{Gahan Saraiya}
\newcommand\thesubject{Compiler Design}
\renewcommand{\footrulewidth}{0.4pt}% default is 0pt
\fancyhead[C]{Institute of Technology, Nirma University $\bullet$ \monthyeardate\today} % Custom header text
\fancyfoot[LE,LO]{\thesubject}
\fancyfoot[RO,LE]{Page \thepage} % Custom footer text
%----------------------------------------------------------------------------------------


\begin{document}
\maketitle % Insert title
\thispagestyle{fancy} % All pages have headers and footers

\section{Introduction}\label{sec:introduction}
Aim of this practical to perform comparison of compiler design tool.
\begin{itemize}
	\item A compiler is a computer program which helps you transform source code written in a high-level language into low-level machine language.
	\item Correctness, speed of compilation, preserve the correct the meaning of the code are some important features of compiler design
	\item Compilers are divided into three parts 1) Single Pass Compilers 2)Two Pass Compilers, and 3) Multipass Compilers
	\item Important compiler construction tools are 1) Scanner generators, 2)Syntax-3) directed translation engines, 4) Parser generators, 5) Automatic code generators
	\item The main task of the compiler is to verify the entire program, so there are no syntax or semantic errors
\end{itemize}

Below are few examples of compiler designer tools:
\begin{itemize}
	\item LEX
	\item Yacc
	\item Parser Generators
	\item Scanner Generators
	\item Syntax Directed Translation engines
\end{itemize}

\renewcommand\arraystretch{2}
\begin{longtable}{p{7cm} | p{7cm}}
	\caption{Comparative study of LEX and Yacc}
	\\
	\textbf{LEX} & \textbf{Yacc}
	\\ \hline
	LEX is lexical analyzer that breaks up an input  stream into usable elements called TOKENS.
	& YACC is a parsing tool that verifies the syntax of input and analyzes the structure.
	\\ \hline
	It deals with tokens
	& It deals with association and precedence.
	\\ \hline
	It consists of set 0 regular expressions.
	& It consists of commands like LALR.
	\\ \hline
	
	Lex is a tool for generating scanners. Scanners are programs that recognize lexical patterns in text.
	& Yacc stands for Yet Another Compiler Compiler.It is a tool that translates any grammar that describes a language into a parser for that language. 
	
	\\ \hline
	
	These lexical patterns (or regular expressions) are defined in a particular syntax. A matched regular expression may have an amciated action
	& It is written in Backus Naur form (BNF).The GNU equivalent of Yacc is called Bison. 
	
	\\ \hline
	
	This action may also include returning a token. 
	& 
	To clarify this concept a bit further, let's take the English language for an example. The set of tokens might be: noun, verb, adjective, and so on. 
	
	\\ \hline
\end{longtable}

\end{document}
